\documentclass{article}

\usepackage{pdfpages}
\usepackage[utf8]{inputenc}
\usepackage{eurosym}
%\usepackage{fancyhdr}
%\pagestyle{fancy}

%
%Die eingegangenen Projektskizzen werden nach folgenden Kriterien bewertet:
%– Zielabdeckung: Das Vorhaben ist auf die Vermittlungsziele des Wissenschaftsjahres 2019 zugeschnitten (Berück-
%sichtigung der Handlungsfelder, Zielgruppen, Forschungsinhalte).
%– Fachliche Kompetenz: Der Antragsteller ist qualifiziert, das Vorhaben durchzuführen und verfügt über nachgewie-
%sene Expertise über das Themenfeld und/oder die Vermittlung des Themenfelds.
%– Schlüssigkeit und Konsistenz des Konzepts: Idee, Ziele, Budgetschätzung.
%– Kommunikative Ausrichtung und Wirksamkeit: Das Vorhaben wird von geeigneten Kommunikationsmaßnahmen be-
%gleitet, es ist öffentlichkeitswirksam und generiert voraussichtlich eine mediale Berichterstattung. Das Vorhaben wird
%kommunikativ in das Wissenschaftsjahr 2018 eingebunden und als Teil dessen wahrgenommen.
%– Innovation: Das Vorhaben ist innovativ und leistet einen Beitrag zur Weiterentwicklung der Wissenschaftskommuni-
%kation in Deutschland.
%– Überregionalität und Nachhaltigkeit: Das Vorhaben strahlt überregional aus und/oder kann übertragen bzw. nachgenutzt werden


\begin{document}


\begin{center}
%\vspace{-13cm}
{ \centering \includegraphics[width=10cm]{hsf.png} }

\vspace{-0.8cm}
\begin{figure*}[h!]
\centering
\includegraphics[width=0.6\textwidth]{arm.png}
\end{figure*}
\vspace{-1.8cm}

{\Huge\bf
PROJEKTSKIZZE: Lernen in Handarbeit
}
\\[2em]
\Large
{\bf Antragsteller:} Prof. Dr. Alexander Gepperth (HDR)\\
{\bf Adresse: } Fachbereich Angewandte Informatik der HAW Fulda\\
Leipzigerstr. 123, 36037 Fulda\\
{\bf Email: } alexander.gepperth@cs.hs-fulda.de\\
{\bf Telefon: } 0661 9640 3485\\
\vspace{0.5cm}

{\bf Projektdauer:} 9 Monate\\
\vspace{0.5cm}

{\bf Gesamtkosten:} 150.000\euro\\
\vspace{0.5cm}

{\bf Zuwendungsbedarf:} 150.000\euro\\
\vspace{0.5cm}

%{\bf Konsortium:} Assisted Living, Hilfe für Schwerbehinderte, Service-Robotik, 3D-Simulation
\end{center}
%\end{titlepage}
\newpage

\renewcommand{\thesection}{2}
\section{Antragsteller und assoziierte Partner}
blablabla

\renewcommand{\thesection}{3}
\section{Ausgangsfrage und Ziele des geplanten Vorhabens}
\subsection{Kurzzusammenfassung}
\subsection{Grundsätzliche Zielsetzung}
%
\renewcommand{\thesection}{4}
\section{Ausführliche Vorhabensbeschreibung}

\renewcommand{\thesection}{5}
\section{Darstellung des Eigeninteresses/Eigenanteils}


\renewcommand{\thesection}{6}
\section{Nachhaltigkeit,Übertragbarkeit}

\renewcommand{\thesection}{7}
\section{Zeitplan}

\renewcommand{\thesection}{8}
\section{Finanzierungsplan}


\renewcommand{\refname}{}
\bibliographystyle{abbrv}
\bibliography{bib.bib}
%

\end{document}




